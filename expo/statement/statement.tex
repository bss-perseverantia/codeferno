\documentclass[12pt,a4paper]{article}

\usepackage[margin=1in]{geometry}
\usepackage{titlesec}
\usepackage{fancyhdr}
\usepackage{listings}
\usepackage{amsmath}
\usepackage{xcolor}
\usepackage{tabularx}

\pagestyle{fancy}
\fancyhf{}
\lhead{Codeferno}
\rhead{Simple Exponentiation}
\cfoot{\thepage}

\titleformat{\section}{\large\bfseries}{\thesection}{1em}{}
\titleformat{\subsection}{\normalsize\bfseries}{\thesubsection}{1em}{}

\begin{document}

\begin{center}
    {\LARGE \textbf{Simple Exponentiation}} \\[0.5em]
    {Problem ID: expo} \\[1em]
    \rule{\textwidth}{0.4pt}
\end{center}

\vspace{1em}

\section*{Problem Statement}
You are given $t$ test cases.  
In each test case, you are given three integers: $a$, $b$, and $m$.  

Your task is to compute $a^b \bmod m$, i.e., the remainder when $a$ raised to the power $b$ is divided by $m$.

\section*{Input}
\begin{itemize}
    \item The first line contains a single integer $t$ ($1 \le t \le 10^5$) — the number of test cases.  
    \item Each of the next $t$ lines contains three space-separated integers $a$, $b$, and $m$ ($1 \le a, b, m \le 10^9$).  
\end{itemize}

\section*{Output}
Print $t$ lines.  
For each test case, output a single integer — the value of $a^b \bmod m$.

\section*{Subtasks}
\begin{center}
\begin{tabularx}{\textwidth}{|c|X|c|}
\hline
\textbf{Subtask} & \textbf{Constraints} & \textbf{Points} \\
\hline
1 & $b \leq 100$ & 10 \\
\hline
2 & $b \leq 10^5$ & 20 \\
\hline
3 & $b$ is a power of $2$ & 30 \\
\hline
4 & No additional constraints & 40 \\
\hline
\end{tabularx}
\end{center}

\section*{Sample Input 1}
\begin{verbatim}
3
2 5 100
3 10 50
10 9 6
\end{verbatim}

\section*{Sample Output 1}
\begin{verbatim}
32
49
4
\end{verbatim}

\section*{Explanation for Sample 1}
\begin{itemize}
  \item $2^5 = 32$, and $32 \bmod 100 = 32$  
  \item $3^{10} = 59049$, and $59049 \bmod 50 = 49$  
  \item $10^9 = 1000000000$, and $1000000000 \bmod 6 = 4$  
\end{itemize}

\vfill
\begin{center}
    \rule{0.8\textwidth}{0.4pt} \\[0.5em]
    \textit{End of Problem 2}
\end{center}

\end{document}