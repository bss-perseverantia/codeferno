\documentclass[12pt,a4paper]{article}

\usepackage[margin=1in]{geometry}
\usepackage{titlesec}
\usepackage{fancyhdr}
\usepackage{listings}
\usepackage{amsmath}
\usepackage{xcolor}
\usepackage{tabularx}

\pagestyle{fancy}
\fancyhf{}
\lhead{Codeferno}
\rhead{Bus Stops}
\cfoot{\thepage}

\titleformat{\section}{\large\bfseries}{\thesection}{1em}{}
\titleformat{\subsection}{\normalsize\bfseries}{\thesubsection}{1em}{}

\begin{document}

\begin{center}
    {\LARGE \textbf{Bus Stops}} \\[0.5em]
    {Problem ID: bus} \\[1em]
    \rule{\textwidth}{0.4pt}
\end{center}

\vspace{1em}

\section*{Problem Statement}
A school bus travels along a straight road with $N$ bus stops, numbered from $1$ to $N$.  
At each stop:
\begin{itemize}
    \item First, some students get off the bus.  
    \item Then, some students get on the bus.  
\end{itemize}

The bus starts empty before stop $1$. Your task is to determine the \textbf{maximum number of students on the bus at any time}.

In some subtasks, additional rules apply (such as bus capacity). Read the constraints carefully.

\section*{Input}
\begin{itemize}
    \item The first line contains an integer $N$ ($1 \leq N \leq 1000$), the number of bus stops.  
    \item The second line contains $N$ integers $\text{on}_1, \text{on}_2, \ldots, \text{on}_N$ ($0 \leq \text{on}_i \leq 10^4$),  
    where $\text{on}_i$ is the number of students boarding at stop $i$.  
    \item The third line contains $N$ integers $\text{off}_1, \text{off}_2, \ldots, \text{off}_N$ ($0 \leq \text{off}_i \leq 10^4$),  
    where $\text{off}_i$ is the number of students getting off at stop $i$.  
    \item The fourth line contains an integer $C$, the bus capacity.  
    For subtasks where no capacity restriction applies, $C$ will be very large (e.g.\ $10^9$), so it does not affect the result.  
\end{itemize}

\section*{Output}
Print a single integer: the maximum number of students on the bus at any point in time.  
In subtasks requiring the stop index, print two integers:  
\texttt{max\_students stop\_index}.

\section*{Subtasks}
\begin{center}
\begin{tabularx}{\textwidth}{|c|X|c|}
\hline
\textbf{Subtask} & \textbf{Constraints} & \textbf{Points} \\
\hline
1 & $N \leq 3$, $\text{on}_i, \text{off}_i \leq 10$, $C = 10^9$ & 10 \\
\hline
2 & $N \leq 100$, $\text{on}_i, \text{off}_i \leq 100$, $C = 10^9$ & 20 \\
\hline
3 & $N \leq 1000$, $\text{on}_i, \text{off}_i \leq 10^4$, $C = 10^9$ & 20 \\
\hline
4 & Same as Subtask 3, but $C \leq 10^4$. If more students attempt to board than seats available, only as many as possible get on. The rest are left behind permanently. & 30 \\
\hline
5 & Same as Subtask 3. Additionally, print the \textbf{first stop index} at which the maximum occupancy occurs. Output format: two integers, \texttt{max\_students stop\_index}. & 20 \\
\hline
\end{tabularx}
\end{center}

\section*{Sample Input 1}
\begin{verbatim}
5
0 3 4 0 2
0 0 2 3 4
1000000000
\end{verbatim}

\section*{Sample Output 1}
\begin{verbatim}
5
\end{verbatim}

\section*{Explanation for Sample 1}
\begin{itemize}
  \item Stop 1: 0 off, 0 on → 0  
  \item Stop 2: 0 off, 3 on → 3  
  \item Stop 3: 2 off, 4 on → 5  
  \item Stop 4: 3 off, 0 on → 2  
  \item Stop 5: 4 off, 2 on → 0  
\end{itemize}

Maximum = 5.

\section*{Sample Input 2 (capacity example)}
\begin{verbatim}
4
5 5 5 5
0 0 0 0
8
\end{verbatim}

\section*{Sample Output 2}
\begin{verbatim}
8
\end{verbatim}

\section*{Explanation for Sample 2}
\begin{itemize}
  \item Stop 1: 5 board → 5  
  \item Stop 2: 5 try to board, but capacity is 8 → 3 board, 2 left behind → total = 8  
  \item Stops 3 and 4: bus remains full at 8  
\end{itemize}

Maximum = 8.

\section*{Sample Input 3 (stop index example)}
\begin{verbatim}
6
2 4 0 2 0 0
0 0 2 0 1 4
1000000000
\end{verbatim}

\section*{Sample Output 3}
\begin{verbatim}
6 2
\end{verbatim}

\section*{Explanation for Sample 3}
\begin{itemize}
  \item Stop 1: 2 board → 2  
  \item Stop 2: 4 board → 6  
  \item Stop 3: 2 off → 4  
  \item Stop 4: 2 on → 6  
  \item Stop 5: 1 off → 5  
  \item Stop 6: 4 off → 1  
\end{itemize}

Maximum = 6 at stop 2, but maximum \textbf{after} that is 6 at stop 4.  
Since the first maximum is at stop 2, output is \texttt{6 2}.  

\vfill
\begin{center}
    \rule{0.8\textwidth}{0.4pt} \\[0.5em]
    \textit{End of Problem 1}
\end{center}

\end{document}
