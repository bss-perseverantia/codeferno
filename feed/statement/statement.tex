\documentclass[12pt,a4paper]{article}

\usepackage[margin=1in]{geometry}
\usepackage{titlesec}
\usepackage{fancyhdr}
\usepackage{listings}
\usepackage{amsmath}
\usepackage{xcolor}
\usepackage{tabularx}

\pagestyle{fancy}
\fancyhf{}
\lhead{Codeferno}
\rhead{Offthentic Feed}
\cfoot{\thepage}

\titleformat{\section}{\large\bfseries}{\thesection}{1em}{}
\titleformat{\subsection}{\normalsize\bfseries}{\thesubsection}{1em}{}

\begin{document}

\ifdefined\BOOKLET
  \pagestyle{empty}
\fi

\begin{center}
    {\LARGE \textbf{Offthentic Feed}} \\[0.5em]
    {Problem ID: feed} \\[1em]
    \rule{\textwidth}{0.4pt}
\end{center}

\vspace{1em}

\section*{Problem Statement}
You are building a new social media platform called \textbf{Offthentic}. Users post content continuously, and the platform must display a live feed of posts.  

There are $n$ posts. Each post $i$ has three attributes:
\begin{itemize}
    \item $u_i$ — the user ID of the author,  
    \item $t_i$ — the timestamp when the post was created (seconds since the start),  
    \item $l_i$ — the number of likes the post has.  
\end{itemize}

The feed has the following rules:
\begin{itemize}
    \item At each second, all posts created up to that time are available.  
    \item The feed displays at most $k$ posts at a time.  
    \item Posts are ranked by:
        \begin{enumerate}
            \item Higher likes ($l_i$),  
            \item Later timestamp ($t_i$),  
            \item Smaller user ID ($u_i$).  
        \end{enumerate}
    \item When a new post enters the feed, it may push out an older one. Once a post leaves the feed, it does not reappear later.  
\end{itemize}

It is guaranteed that no two posts have identical $(u_i, t_i)$ pairs, so ties are always resolvable.  

Your task is to simulate the feed and output the order in which posts first appeared.  
\textbf{Note:} Some posts may never appear if they are always out-ranked.

\section*{Input}
\begin{itemize}
    \item The first line contains two integers $n$ and $k$ ($1 \leq n, k \leq 10^5$).  
    \item The next $n$ lines each contain three integers $u_i$, $t_i$, and $l_i$ ($1 \leq u_i \leq 10^9$, $0 \leq t_i \leq 10^9$, $0 \leq l_i \leq 10^5$).  
\end{itemize}

\section*{Output}
Print the indices of the posts (1-based, in input order) in the order they \textbf{first appear} in the feed.  
If a post never appears, it should not be printed.

\section*{Subtasks}
\begin{center}
\begin{tabularx}{\textwidth}{|c|X|c|}
\hline
\textbf{Subtask} & \textbf{Constraints} & \textbf{Points} \\
\hline
1 & $n \leq 100$, $k = 1$ & 20 \\
\hline
2 & $n \leq 2000$, $k \leq 10$ & 20 \\
\hline
3 & $n \leq 10^5$, $k \leq 100$ & 30 \\
\hline
4 & No additional constraints & 30 \\
\hline
\end{tabularx}
\end{center}

\section*{Sample 1}
\subsection*{Input}
\begin{verbatim}
5 2
10 1 5
7 2 5
3 2 8
5 3 5
2 5 10
\end{verbatim}

\subsection*{Output}
\begin{verbatim}
1 3 2 4 5
\end{verbatim}

\subsection*{Explanation}
\begin{itemize}
    \item At $t=1$, post 1 appears.  
    \item At $t=2$, posts 2 and 3 are available. Post 3 has more likes, so it enters; post 2 also enters while post 1 is pushed out.  
    \item At $t=3$, post 4 joins.  
    \item At $t=5$, post 5 (10 likes) pushes out post 2.  
\end{itemize}

The order of first appearance is: 1, 3, 2, 4, 5.  

\section*{Sample 2}
\subsection*{Input}
\begin{verbatim}
4 2
1 1 100
2 2 50
3 3 100
4 4 200
\end{verbatim}

\subsection*{Output}
\begin{verbatim}
1 2 3 4
\end{verbatim}

\subsection*{Explanation}
\begin{itemize}
    \item At $t=1$, post 1 appears.  
    \item At $t=2$, post 2 enters with lower likes but fills the empty slot.  
    \item At $t=3$, post 3 (likes $=100$) replaces post 2 (likes $=50$).  
    \item At $t=4$, post 4 (likes $=200$) pushes out post 1.  
\end{itemize}

Final order: 1, 2, 3, 4. Post 2 never reappears after being pushed out, even though newer posts arrive.  

\vfill
\begin{center}
    \rule{0.8\textwidth}{0.4pt} \\[0.5em]
    \textit{End of Problem 3}
\end{center}

\end{document}
