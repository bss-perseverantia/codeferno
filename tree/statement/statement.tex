\documentclass[12pt,a4paper]{article}

\usepackage[margin=1in]{geometry}
\usepackage{titlesec}
\usepackage{fancyhdr}
\usepackage{listings}
\usepackage{amsmath}
\usepackage{xcolor}
\usepackage{tabularx}

\pagestyle{fancy}
\fancyhf{}
\lhead{Codeferno}
\rhead{Binary Tree Traversal}
\cfoot{\thepage}

\titleformat{\section}{\large\bfseries}{\thesection}{1em}{}
\titleformat{\subsection}{\normalsize\bfseries}{\thesubsection}{1em}{}

\begin{document}

\begin{center}
    {\LARGE \textbf{Binary Tree Traversal}} \\[0.5em]
    {Problem ID: tree} \\[1em]
    \rule{\textwidth}{0.4pt}
\end{center}

\vspace{1em}

\section*{Problem Statement}
You are given a \textbf{binary tree} with $n$ nodes, rooted at node $1$.  
The tree is specified using edges between nodes.  

Each node has a unique integer label from $1$ to $n$.  
Your task is to output the \textbf{inorder}, \textbf{preorder}, and \textbf{postorder} traversals of this tree.

\section*{Traversal Definitions}
\begin{itemize}
  \item \textbf{Preorder}: Visit the root, then recursively visit the left subtree, then the right subtree.  
  \item \textbf{Inorder}: Recursively visit the left subtree, then the root, then the right subtree.  
  \item \textbf{Postorder}: Recursively visit the left subtree, then the right subtree, then the root.  
\end{itemize}

If a node has no left or right child, skip that direction during traversal.  
When a node has two children, the \textbf{smaller numbered child is the left child}.  

\section*{Input}
\begin{itemize}
    \item The first line contains a single integer $n$ ($1 \le n \le 10^5$) — the number of nodes.  
    \item Each of the next $n-1$ lines contains two space-separated integers $u$ and $v$ ($1 \le u, v \le n$), representing an edge between nodes $u$ and $v$.  
\end{itemize}

The input is guaranteed to form a valid binary tree rooted at node $1$.

\section*{Output}
Print three lines:
\begin{itemize}
  \item The \textbf{inorder traversal} of the tree (space-separated).  
  \item The \textbf{preorder traversal} of the tree.  
  \item The \textbf{postorder traversal} of the tree.  
\end{itemize}

Each line should contain exactly $n$ integers representing the node labels in the corresponding traversal order.

\section*{Subtasks}
\begin{center}
\begin{tabularx}{\textwidth}{|c|X|c|}
\hline
\textbf{Subtask} & \textbf{Constraints} & \textbf{Points} \\
\hline
1 & $n = 1$ & 5 \\
\hline
2 & $n = 3$ & 15 \\
\hline
3 & No additional constraints & 80 \\
\hline
\end{tabularx}
\end{center}

\section*{Sample Input 1}
\begin{verbatim}
3
1 2
1 3
\end{verbatim}

\section*{Sample Output 1}
\begin{verbatim}
2 1 3
1 2 3
2 3 1
\end{verbatim}

\section*{Explanation for Sample 1}
The tree looks like:
\begin{verbatim}
    1
   / \
  2   3
\end{verbatim}

\begin{itemize}
  \item \textbf{Inorder}: Left → Root → Right → $2\ 1\ 3$  
  \item \textbf{Preorder}: Root → Left → Right → $1\ 2\ 3$  
  \item \textbf{Postorder}: Left → Right → Root → $2\ 3\ 1$  
\end{itemize}

\vfill
\begin{center}
    \rule{0.8\textwidth}{0.4pt} \\[0.5em]
    \textit{End of Problem 3}
\end{center}

\end{document}