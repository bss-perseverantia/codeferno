\documentclass[12pt,a4paper]{article}

\usepackage[margin=1in]{geometry}
\usepackage{titlesec}
\usepackage{fancyhdr}
\usepackage{listings}
\usepackage{xcolor}

\pagestyle{fancy}
\fancyhf{}
\lhead{Codeferno}
\rhead{Perseverantia 2025}
\cfoot{\thepage}

\lstset{
  basicstyle=\ttfamily\small,
  frame=single,
  breaklines=true,
  showstringspaces=false,
  tabsize=2,
  numbers=left,
  numberstyle=\tiny,
  keywordstyle=\color{blue},
  commentstyle=\color{gray},
  stringstyle=\color{brown}
}

\begin{document}

\begin{center}
    {\LARGE \textbf{Sample Problem Statement}} \\[0.5em]
    \rule{\textwidth}{0.4pt}
\end{center}

\vspace{1em}

\section*{Introduction}
This document contains an example of a problem that might appear in Codeferno.  
It is provided to help you get familiar with the problem format, input/output style, subtasks, and solutions in different programming languages (C, C++, Python, and Java).  

\section*{Problem A: Sum of Two Numbers}
You are given two integers $a$ and $b$. Your task is to compute their sum.

\subsection*{Input}
Two integers $a$ and $b$.

\subsection*{Output}
Output a single integer, the value of $a+b$.

\subsection*{Constraints}
\begin{tabular}{|c|c|c|}
\hline
\textbf{Subtask} & \textbf{Constraints} & \textbf{Points} \\
\hline
1 & $-10^9 \leq a, b \leq 10^9$ & 30 \\
\hline
2 & $-10^{18} \leq a, b \leq 10^{18}$ & 70 \\
\hline
\end{tabular}

\subsection*{Sample Input}
\begin{verbatim}
5 7
\end{verbatim}

\subsection*{Sample Output}
\begin{verbatim}
12
\end{verbatim}

\section*{Solutions}

\subsection*{C (filename: any .c file)}
\begin{lstlisting}[language=C]
#include <stdio.h>

int main() {
  long long a, b;
  scanf("%lld %lld", &a, &b);
  printf("%lld\n", a + b);
}
\end{lstlisting}

\subsection*{C++ (filename: any .cpp file)}
\begin{lstlisting}[language=C++]
#include <bits/stdc++.h>

using namespace std;

int main() {
  long long a, b;
  cin >> a >> b;
  cout << a + b << "\n";
}
\end{lstlisting}

\subsection*{Python (filename: any .py file)}
\begin{lstlisting}[language=Python]
a, b = map(int, input().split())
print(a + b)
\end{lstlisting}

\subsection*{Java (filename: A.java)}
\begin{lstlisting}[language=Java]
import java.util.*;

public class A {
  public static void main(String[] args) {
    Scanner sc = new Scanner(System.in);
    long a = sc.nextLong();
    long b = sc.nextLong();
    System.out.println(a + b);
  }
}
\end{lstlisting}

\section*{Notes}
\begin{itemize}
    \item In C, C++, and Python, the filename does not matter.  
    \item In Java, the filename must match the problem ID.  
For example, for this problem (Problem A), the file should be named \texttt{A.java}.  
    \item Always follow the input/output format strictly. Do not add prompts or extra text.  
\end{itemize}

\end{document}
